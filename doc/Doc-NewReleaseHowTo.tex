% !TEX encoding = UTF-8 Unicode
% !TEX spellcheck = en_US

\documentclass[a4paper, 11pt]{article}

\usepackage[utf8]{inputenc}
\usepackage[T1]{fontenc}
\usepackage{lmodern}
\usepackage[margin=2.4cm]{geometry} % warning: space added line 238. 
\usepackage{enumitem}
\usepackage{fancyhdr}
\usepackage{lastpage}
\usepackage{hyperref}
\usepackage{color}
\usepackage{indentfirst}

% Specific commands
\newcommand{\FullSWOFnD}{\emph{FullSWOF\_1D/2D}}
\newcommand{\FullSWOFoneD}{\emph{FullSWOF\_1D}}
\newcommand{\FullSWOFtwoD}{\emph{FullSWOF\_2D}}
\newcommand{\FullSWOFUI}{\emph{FullSWOF\_UI}}
\newcommand{\SWASHES}{\emph{SWASHES}}
\newcommand{\contactFullName}{Frédéric \textsc{Darboux}}
\newcommand{\contactEmail}{\href{mailto:Frederic.Darboux@orleans.inra.fr}{Frederic.Darboux@orleans.inra.fr}}
\newcommand{\contactFS}{\href{mailto:fullswof.contact@listes.univ-orleans.fr}{fullswof.contact@listes.univ-orleans.fr}}
\newcommand{\contactSWASHES}{\href{mailto:swashes.contact@listes.univ-orleans.fr}{swashes.contact@listes.univ-orleans.fr}}
\newcommand{\MainWebSiteFSoneD}{\url{https://sourcesup.renater.fr/projects/fullswof-1d/}}
\newcommand{\MainWebSiteFStwoD}{\url{https://sourcesup.renater.fr/projects/fullswof-2d/}}
\newcommand{\MainWebSiteFSUI}{\url{https://sourcesup.renater.fr/projects/fullswof-ui/}}
\newcommand{\MainWebSiteSWASHES}{\url{https://sourcesup.renater.fr/projects/swashes/}}

\renewcommand{\thesection}{\thepart .\arabic{section}}

% Headers and footers
\lhead{Releasing \FullSWOFnD{}, \FullSWOFUI{} or \SWASHES{}}
\rhead{INRA-UR272-Sols --- MAPMO}
\cfoot{\thepage /\pageref{LastPage}}
\pagestyle{fancy}

\def\thepart{\Alph{part}}

% Lists: no spacing between items
\setlist{noitemsep}

\title{Documentation:\\
Steps to release a new version of\\
\FullSWOFoneD, \FullSWOFtwoD{}, \FullSWOFUI{} or \SWASHES{}.\\
General release and Windows release}

\author{\contactFullName, \contactEmail}
\date{2014-04-29}

\begin{document}

\maketitle

\thispagestyle{fancy}

This document lists the steps to release a new version of \FullSWOFoneD, \FullSWOFtwoD{}, \FullSWOFUI{} or \SWASHES{}. The first part is related to the general release, \emph{i.e.}~the non-OS-specific packages. The second part describes the release of packages dedicated to Windows.

For any question, please contact us at the following addresses: \contactFS{} or \contactSWASHES{}.

\part{For general release}

Releasing non-OS-specific packages for a new version of \FullSWOFoneD, \FullSWOFtwoD{}, \FullSWOFUI{} or \SWASHES{} requires a succession of operations using various tools. 
This part details these steps.


\section{From trunk to tag}

\emph{Remark: it is assumed that you have already downloaded the subdirectories named trunk, tags and branches from the repository.}

After thorough checks (esp. compilation and run of the benchmarks under various operating systems), it can be decided to release a new version of the software. A final revision change should be committed to SVN. It consists in: 
\begin{enumerate}
\item adding to the file trunk/changelog.txt something like:
\begin{verbatim}
2013-09-04
READY FOR TAG version 1.04.04
\end{verbatim}
\item running the script UpdateDateVersion.sh  from the trunk directory with parameters such as
\begin{verbatim}
bin/UpdateDateVersion.sh 2013-09-04 1.04.04
\end{verbatim}
\item committing these changes to the SVN server:
\begin{verbatim}
svn commit -m "Ready for tag version 1.04.04"
\end{verbatim}
\item creating the tag:
\begin{verbatim}
svn copy https://subversion.renater.fr/fullswof-2d/trunk \
         https://subversion.renater.fr/fullswof-2d/tags/release-1.04.04 \
      -m "Tagging the 1.04.04 release of the FullSWOF_2D project."
\end{verbatim}
\end{enumerate}


\section{From tag to package}

\begin{enumerate}
\item Move in the tags directory and do an update:
\begin{verbatim}
 cd tags/
 svn update
\end{verbatim}
\item copy the relevant tag directory and give it the corresponding name:
\begin{verbatim}
mkdir FullSWOF_2D-1.04.04
cp -r release-1.04.04/* FullSWOF_2D-1.04.04
\end{verbatim}
\item remove the .svn directories
\begin{verbatim}
find FullSWOF_2D-1.04.04 -name ".svn" -type d -exec rm -rf {} \;
\end{verbatim}

\item create the zip file for \FullSWOFoneD, \FullSWOFtwoD{},\FullSWOFUI{} and \SWASHES{}
\begin{verbatim}
zip -r FullSWOF_2D-1.04.04.zip FullSWOF_2D-1.04.04
\end{verbatim}
\item create a text file (\emph{e.g.}~comment.txt) with some info such as 
\begin{verbatim}
FullSWOF_2D release 1.04.04
<https://sourcesup.renater.fr/projects/fullswof-2d/>
\end{verbatim}
\item add this info to the zip file
\begin{verbatim}
zip --archive-comment FullSWOF_2D-1.04.04.zip < comment.txt
\end{verbatim}

\color[gray]{0.4}
\item in the case of \FullSWOFUI{}, create the jar file 

\begin{enumerate}
 
\item import the directory (\emph{e.g.}, FullSWOF\_UI-1.01.00) into Eclipse.
\item create the archive (.jar file)

\begin{enumerate}
\item go to the menu ``File->Export'', select ``Runnable JAR file'' and click on ``Next''.
\item in the field ``Export Destination'', put the name of the .jar file (\emph{e.g.} FullSWOF\_UI-v1.01.00.jar).
\item in ``Library handling'', select ``Package required libraries into generated JAR''.
\item click on ``Finish''.
\end{enumerate}

\item add the two licence files to the .jar file

\begin{verbatim}
zip FullSWOF_UI-1.01.00.jar FullSWOF_UI-1.01.00/LICENSE_en.txt \
FullSWOF_UI-1.01.00/LICENSE_fr.txt
\end{verbatim}

\item check the .jar file is running properly
\begin{verbatim}
java -jar FullSWOF_UI-1.01.00.jar
\end{verbatim}

\end{enumerate} 

\color{black}

\item remove the file directory
\begin{verbatim}
rm -rf FullSWOF_2D-1.04.04
\end{verbatim}
\end{enumerate}


\section{From upload to messages}
\begin{enumerate}
\item Release the file:
\begin{enumerate}
\item Log in the SourceSup website and go to the project tab ``Files''.
\item Click on ``To create a new release click here.''
\item Fill the fields, \emph{e.g.}:\\
\emph{Remark: for \FullSWOFUI, compatibility with \FullSWOFoneD{} and \FullSWOFtwoD{} should be specified.}
\begin{verbatim}
Release name: 1.04.04
File Name: <Select the proper file>
File type: Source .zip
Processor type: Any
Release Notes:
     FullSWOF_2D release 1.04.04
     This is a major update introducing new features.
     It also include bug fixes.
Change Log:
     Main changes:
       - add of infiltration
       - add of rain
     For details see file changelog.txt
\end{verbatim}
\item Click on ``Release file''.
\end{enumerate}

\color[gray]{0.4}
\item In case of \FullSWOFUI{}, add the jar file in the Release file:
\begin{enumerate}
\item Go to the project tab ``Files''.
\item In the ``Project Filelist'' page, click on ``Administration.''
\item In the ``Releases'' part, click on ``[Edit Releases]''.
\item Click on ``[Edit]'' for the corresponding release.
\item In the part ``Add Files To This Release''.
\begin{verbatim}
File Name: <Select the proper file>
File type: .jar
Processor type: Any
\end{verbatim}
\item Click on ``Add This File''.

\end{enumerate}

\color{black}

\item Add documents on the web site:
\begin{enumerate}
\item Copy the corresponding documentation (\emph{e.g.} Documentation.pdf to Documentation-v1.04.04.pdf) and add it to the documents of Sourcesup, in the doc folder.
\item Copy the corresponding documentation for developers (\emph{e.g.} refman.pdf to refman-v1.04.04.pdf) and add it to the documents of Sourcesup, in the doc folder. 
\end{enumerate}
\item Post news:
\begin{enumerate}
\item Go to the project tab ``News''.
\item Click on ``Submit''.
\item Add a subject, such as:
\begin{verbatim}
Release of FullSWOF_2D 1.04.04 
\end{verbatim}
\item Add some details in the panel, such as:\\
\emph{Remark: for \FullSWOFUI, compatibility with \FullSWOFoneD{} and \FullSWOFtwoD{} should be specified.}
\begin{verbatim}
The FullSWOF development team is pleased to announce
the release of FullSWOF_2D 1.04.04.
This is just a minor bug release:
the previous release (1.04.03) did not compile under windows (cygwin).
No other changes.
\end{verbatim}
\item Click on ``Submit''.
\end{enumerate}
\item Send an email to the relevant list (\href{mailto:fullswof.infos@listes.univ-orleans.fr}{fullswof.infos@listes.univ-orleans.fr}
or
\href{mailto:swashes.infos@listes.univ-orleans.fr}{swashes.infos@listes. univ-orleans.fr}), such as:\\
\emph{Remark: for \FullSWOFUI, compatibility with \FullSWOFoneD{} and \FullSWOFtwoD{} should be specified.}
\begin{verbatim}
Title: FullSWOF_2D : release 1.04.04 : bug fix (compilation pb)

Dear FullSWOF enthusiasts,

I have just released a new version of FullSWOF_2D (release 1.04.04).
It can be downloaded here:
<https://sourcesup.renater.fr/frs/shownotes.php?release_id=2430>

This is just a minor bug fix: FullSWOF_2D release 1.04.03 did not
compile under windows cygwin (Bug #8814).

If you are using the release 1.04.03 and if it compiled smoothly
under your system, you do not have to worry about it. Just keep working!

Cheers.

The FullSWOF development team 
\end{verbatim}
\end{enumerate}

\part{For Windows release}
\setcounter{section}{0}

Releasing \emph{Windows}-specific packages for a new version of \FullSWOFoneD, \FullSWOFtwoD{}, \FullSWOFUI{} or \SWASHES{} requires a succession of operations using various tools. 
This part details these steps.


\section{\SWASHES{}}

\subsection{From zip to \_win.zip}
\begin{enumerate}
\item Uncompress the general package of \SWASHES{} (\emph{i.e.} the zip file)
\item In the file ``Makefile'', change the first lines into
\begin{verbatim}
CC=i686-w64-mingw32-g++

SRCDIR=Sources
HEADDIR=Headers
OBJDIR=obj
BINDIR=bin

CFLAGS= -I$(HEADDIR) -Wall -pedantic -O3 -static

EXEC=swashes_win.exe
\end{verbatim}
\item Start cygwin and go to the main directory.
\item Launch ``make''.\\
In bin/, you should now have ``swashes\_win.exe''
\item Uncompress the original zip file and add ``\_win'' to the directory name.
\item Create the \_win.zip file
\begin{verbatim}
zip -r SWASHES-1.01.04_win.zip SWASHES-1.01.04_win
\end{verbatim}
\item Inside this new zip file, add to bin/
	\begin{itemize}
		\item CommandLinePrompt.lnk
		\item swashes\_win.exe
	\end{itemize}
\item Add to ./ an updated version of the file README\_win.txt.
\item Create a text file (\emph{e.g.} comment.txt) with some info such as
\begin{verbatim}
SWASHES release 1.01.04 for Windows
<https://sourcesup.renater.fr/projects/swashes/>
\end{verbatim}
\item Add this info to the zip file
\begin{verbatim}
zip --archive-comment SWASHES-1.01.04_win.zip < comment.txt
\end{verbatim}
\item Remove the file directory
\begin{verbatim}
rm -rf SWASHES-1.01.04_win
\end{verbatim}

\item Test the \_win.zip file under winxp, win7.32 and win7.64 by
double-clicking on CommandLinePrompt.lnk and launching
\begin{verbatim}
swashes_win.exe 2 1 1 2 100 200 > result.txt
\end{verbatim}
\end{enumerate}

\subsection{From upload to messages}
\begin{enumerate}
\item Release the \_win.zip file:
\begin{enumerate}
\item Log in the SourceSup website and go to the project tab ``Files''.
\item Click on ``Administration''.
\item Click on ``[Edit Releases]'' (at the bottom).
\item Click on the ``[Edit]'' of the relevant release.
\item In the field ``Paste The Notes In:'', add something like:
\begin{verbatim}
A Windows exe is also available in a specific package.
It can run on winxp, win7 32 bits and win 7 64 bits.
For details see file README_win.txt in the _win.zip file.
\end{verbatim}
\item Click on ``Submit/Refresh''.
\item In the section ``Add Files To This Release'', select the \_win.zip file,
the file type (``.zip'') and the processor type  (``i386''). Click on ``Add this file''.
\end{enumerate}
\item Post news:
\begin{enumerate}
\item Go to the project tab ``News''.
\item Click on ``Submit''.
\item Add a subject, such as:
\begin{verbatim}
Windows release of SWASHES 1.01.04 
\end{verbatim}
\item Add some details in the panel, such as:
\begin{verbatim}
The SWASHES development team is pleased to announce
the release of the Windows package of SWASHES 1.01.04.
It is similar to the general package, but contains a Windows executable 
able to run under winxp, win7 32 bits and win 7 64 bits.
\end{verbatim}
\item Click on ``Submit''.
\end{enumerate}
\item Send an email to the list
\href{mailto:swashes.infos@listes.univ-orleans.fr}{swashes.infos@listes.univ-orleans.fr}, such as:
\begin{verbatim}
Title: SWASHES: Release 1.01.04 for Windows

Dear SWASHES enthusiasts,

We have just released a version of SWASHES 1.01.04 for Windows.
It can be downloaded here: <https://sourcesup.renater.fr/frs/?group_id=876> 
(file SWASHES-1.01.04_win.zip)

It is designed to help Windows users to run SWASHES on their exploitation 
system. The package contains an executable able to run under winxp, 
win7 32 bits and win 7 64 bits.

For details see file README_win.txt .

Cheers.

The SWASHES development team 
\end{verbatim}
\end{enumerate}

% % % % % % % % % % % % % % % % % % % % % % % % % % % % % % % % % % % % % % % %

\section{\FullSWOFoneD{}}

\subsection{From zip to \_win.zip}
\begin{enumerate}
\item Uncompress the general package of \FullSWOFoneD{} (\emph{i.e.} the zip file).
\item In the file ``make\_config'', change the line
\begin{verbatim}
CPP	= g++
\end{verbatim}
into
\begin{verbatim}
CPP	= i686-w64-mingw32-g++
\end{verbatim}
\item In the file ``Makefile'', change the line
\begin{verbatim}
TARGET := FullSWOF_1D
\end{verbatim}
into
\begin{verbatim}
TARGET := FullSWOF_1D_win.exe
\end{verbatim}
and add
\begin{verbatim}
-static
\end{verbatim}
at the end of the line
\begin{verbatim}
LIBRARY := -lschemes -lboundaryconditions [...] -lparameters -lparser
\end{verbatim}
%\begin{verbatim}
%LIBRARY := -lschemes -lboundaryconditions -lflux -lfrictions -lrain_infiltration -linitializations -lreconstructions -llimitations -lsave -lparameters -lparser
%\end{verbatim}
\item Start cygwin and go to the main directory.
\item Launch ``make''.\\
In bin/, you should now have ``FullSWOF\_1D\_win.exe''.
\item Uncompress the original zip file and add ``\_win'' to the directory name.
\item Create the \_win.zip file
\begin{verbatim}
zip -r FullSWOF_1D-1.01.00_win.zip FullSWOF_1D-1.01.00_win
\end{verbatim}
\item Inside this new zip file, add to Exp01/
	\begin{itemize}
		\item CommandLinePrompt.lnk
		\item FullSWOF\_1D\_win.exe
	\end{itemize}
\item Add to ./ an updated version of the file README\_win.txt.
\item Create a text file (\emph{e.g.} comment.txt) with some info such as
\begin{verbatim}
FullSWOF_1D release 1.01.00 for Windows
<https://sourcesup.renater.fr/projects/fullswof-1d/>
\end{verbatim}
\item Add this info to the zip file
\begin{verbatim}
zip --archive-comment FullSWOF_1D-1.01.00_win.zip < comment.txt
\end{verbatim}
\item Remove the file directory
\begin{verbatim}
rm -rf FullSWOF_1D-1.01.00_win
\end{verbatim}

\item Test the \_win.zip file under winxp, win7.32 and win7.64 by
double-clicking on CommandLinePrompt.lnk and launching
\begin{verbatim}
FullSWOF_1D_win.exe
\end{verbatim}
\end{enumerate}

\subsection{From upload to messages}
\begin{enumerate}
\item Release the \_win.zip file:
\begin{enumerate}
\item Log in the SourceSup website and go to the project tab ``Files''.
\item Click on ``Administration''.
\item Click on ``[Edit Releases]'' (at the bottom).
\item Click on the ``[Edit]'' of the relevant release.
\item In the field ``Paste The Notes In:'', add something like:
\begin{verbatim}
A Windows exe is also available in a specific package 
for EVALUATION PURPOSE ONLY.
It can run on winxp, win7 32 bits and win 7 64 bits.
For details see file README_win.txt in the _win.zip file.
\end{verbatim}
\item Click on ``Submit/Refresh''.
\item In the section ``Add Files To This Release'', select the \_win.zip file,
the file type (``.zip'') and the processor type  (``i386''). Click on ``Add this file''.
\end{enumerate}
\item Post news:
\begin{enumerate}
\item Go to the project tab ``News''.
\item Click on ``Submit''.
\item Add a subject, such as:
\begin{verbatim}
Windows release of FullSWOF_1D 1.01.00
\end{verbatim}
\item Add some details in the panel, such as:
\begin{verbatim}
The FullSWOF development team is pleased to announce
the release of the Windows package of FullSWOF_1D 1.01.00.
It is similar to the general package, but contains a Windows executable 
able to run under winxp, win7 32 bits and win 7 64 bits.
It is distributed for EVALUATION PURPOSE ONLY.
\end{verbatim}
\item Click on ``Submit''.
\end{enumerate}
\item Send an email to the list
\href{mailto:fullswof.infos@listes.univ-orleans.fr}{fullswof.infos@listes.univ-orleans.fr}), such as:
\begin{verbatim}
Title: FullSWOF_1D: Release 1.01.00 for Windows

Dear FullSWOF_1D enthusiasts,

We have just released a version of FullSWOF_1D 1.01.00 for Windows.
It can be downloaded here: <https://sourcesup.renater.fr/frs/?group_id=877>
(file FullSWOF_1D-1.02.00_win.zip)

It is designed to help Windows users to run FullSWOF_1D on their 
exploitation system. The package contains an executable able to run  
under winxp, win7 32 bits and win 7 64 bits.
It is distributed for EVALUATION PURPOSE ONLY.

For details see file README_win.txt .

Cheers.

The FullSWOF development team 
\end{verbatim}
\end{enumerate}


% % % % % % % % % % % % % % % % % % % % % % % % % % % % % % % % % % % % % % % %

\section{\FullSWOFtwoD{}}

\subsection{From zip to \_win.zip}
\begin{enumerate}
\item Uncompress the general package of \FullSWOFtwoD{} (\emph{i.e.} the zip file).
\item In the file ``make\_config'', change the line
\begin{verbatim}
CPP	= g++
\end{verbatim}
into
\begin{verbatim}
CPP	= i686-w64-mingw32-g++
\end{verbatim}
\item In the file ``Makefile'', change the line
\begin{verbatim}
TARGET := FullSWOF_2D
\end{verbatim}
into
\begin{verbatim}
TARGET := FullSWOF_2D_win.exe
\end{verbatim}
and add
\begin{verbatim}
-static
\end{verbatim}
at the end of the line
%\begin{verbatim}
%LIBRARY := -lschemes -lboundaryconditions -lflux -lfrictions -lrain_infiltration -linitializations -lreconstructions -llimitations -lsave -lparameters -lparser
%\end{verbatim}
\begin{verbatim}
LIBRARY := -lschemes -lboundaryconditions [...] -lparameters -lparser
\end{verbatim}
\item Start cygwin and go to the main directory.
\item Launch ``make''.\\
In bin/, you should now have ``FullSWOF\_2D\_win.exe''.
\item Uncompress the original zip file and add ``\_win'' to the directory name.
\item Create the \_win.zip file
\begin{verbatim}
zip -r FullSWOF_2D-1.04.04_win.zip FullSWOF_2D-1.04.04_win
\end{verbatim}
\item Inside this new zip file, add to Exp01/
	\begin{itemize}
		\item CommandLinePrompt.lnk
		\item FullSWOF\_2D\_win.exe
	\end{itemize}
\item Add to ./ an updated version of the file README\_win.txt.
\item create a text file (\emph{e.g.} comment.txt) with some info such as
\begin{verbatim}
FullSWOF_2D release 1.04.04 for Windows
<https://sourcesup.renater.fr/projects/fullswof-2d/>
\end{verbatim}
\item Add this info to the zip file
\begin{verbatim}
zip --archive-comment FullSWOF_2D-1.04.04_win.zip < comment.txt
\end{verbatim}
\item Remove the file directory
\begin{verbatim}
rm -rf FullSWOF_2D-1.04.04_win
\end{verbatim}

\item Test the \_win.zip file under winxp, win7.32 and win7.64 by
double-clicking on CommandLinePrompt.lnk and launching
\begin{verbatim}
FullSWOF_2D_win.exe
\end{verbatim}
\end{enumerate}

\subsection{From upload to messages}
\begin{enumerate}
\item Release the \_win.zip file:
\begin{enumerate}
\item Log in the SourceSup website and go to the project tab ``Files''.
\item Click on ``Administration''.
\item Click on ``[Edit Releases]'' (at the bottom).
\item Click on the ``[Edit]'' of the relevant release.
\item In the field ``Paste The Notes In:'', add something like:
\begin{verbatim}
A Windows exe is also available in a specific package 
for EVALUATION PURPOSE ONLY.
It can run on winxp, win7 32 bits and win 7 64 bits.
For details see file README_win.txt in the _win.zip file.
\end{verbatim}
\item Click on ``Submit/Refresh''.
\item In the section ``Add Files To This Release'', select the \_win.zip file,
the file type (``.zip'') and the processor type  (``i386''). Click on ``Add this file''.
\end{enumerate}
\item Post news:
\begin{enumerate}
\item Go to the project tab ``News''.
\item Click on ``Submit''.
\item Add a subject, such as:
\begin{verbatim}
Windows release of FullSWOF_2D 1.04.04
\end{verbatim}
\item Add some details in the panel, such as:
\begin{verbatim}
The FullSWOF development team is pleased to announce
the release of the Windows package of FullSWOF_2D 1.04.04.
It is similar to the general package, but contains a Windows executable 
able to run under winxp, win7 32 bits and win 7 64 bits.
It is distributed for EVALUATION PURPOSE ONLY.
\end{verbatim}
\item Click on ``Submit''.
\end{enumerate}
\item Send an email to the list
\href{mailto:fullswof.infos@listes.univ-orleans.fr}{fullswof.infos@listes.univ-orleans.fr}, such as:
\begin{verbatim}
Title: FullSWOF_2D: Release 1.04.04 for Windows

Dear FullSWOF_2D enthusiasts,

We have just released a version of FullSWOF_2D 1.04.04 for Windows.
It can be downloaded here: <https://sourcesup.renater.fr/frs/?group_id=895>
(file FullSWOF_2D-1.04.04_win.zip)

It is designed to help Windows users to run FullSWOF_2D on their 
exploitation system. The package contains an executable able to run 
under winxp, win7 32 bits and win 7 64 bits.
It is distributed for EVALUATION PURPOSE ONLY.

For details see file README_win.txt .

Cheers.

The FullSWOF development team 
\end{verbatim}
\end{enumerate}

% % % % % % % % % % % % % % % % % % % % % % % % % % % % % % % % % % % % % % % %

\section{\FullSWOFUI{}}


\subsection{From .jar  to \_win.zip}
\begin{enumerate}

\item Create a directory named FullSWOF\_UI-1.01.00\_win .

\item Copy the jar file of \FullSWOFUI{} into it.

\item Unpack the FullSWOF\_1D\_win.zip and FullSWOF\_2D\_win.zip files into it.

\item In the FullSWOF\_1D\_win and FullSWOF\_2D\_win directories, delete the make\_config and Makefile files, and all the sub-directories except Exp01/ .

\item Add to ./ the files 
FullSWOF\_UI-1.01.00/util/User\_Manual\_en.pdf and
FullSWOF\_UI-1.01.00/util/User\_Manual\_fr.pdf.

\item Add to ./ the two license files FullSWOF\_UI-1.01.00/LICENSE\_en.txt and
FullSWOF\_UI-1.01.00/LICENSE\_fr.txt.

\item Add to ./ an updated version of the file README\_win.txt.

\item Create the zip file
\begin{verbatim}
zip -r FullSWOF_UI-1.01.00_win.zip FullSWOF_UI-1.01.00_win
\end{verbatim}
\item Create a text file (\emph{e.g.} comment.txt) with some info such as
\begin{verbatim}
FullSWOF_UI release 1.01.00 for Windows
<https://sourcesup.renater.fr/projects/fullswof-ui/>
\end{verbatim}
\item Add this info to the zip file
\begin{verbatim}
zip --archive-comment FullSWOF_UI-1.01.00_win.zip < comment.txt
\end{verbatim}
\item Remove the file directory
\begin{verbatim}
rm -rf FullSWOF_UI-1.01.00_win
\end{verbatim}
\item Test the \_win.zip file under winxp, win7.32 and win7.64 by
double-clicking on the .jar file and launching FullSWOF\_1D and FullSWOF\_2D computations.
\end{enumerate}


\subsection{From upload to messages}
\begin{enumerate}
\item Release the \_win.zip file:
\begin{enumerate}
\item Log in the SourceSup website and go to the project tab ``Files''.
\item Click on ``Administration''.
\item Click on ``[Edit Releases]'' (at the bottom).
\item Click on the ``[Edit]'' of the relevant release.
\item In the field ``Paste The Notes In:'', add something like:
\begin{verbatim}
A Windows package is also available.
This package is designed to help you test FullSWOF under Windows.
It is provided for EVALUATION PURPOSE ONLY.
It can run on winxp, win7 32 bits and win 7 64 bits.
For details see file README_win.txt in the _win.zip file.
\end{verbatim}
\item Click on ``Submit/Refresh''.
\item In the section ``Add Files To This Release'', select the \_win.zip file,
the file type (``.zip'') and the processor type  (``i386''). Click on ``Add this file''.
\end{enumerate}
\item Post news:
\begin{enumerate}
\item Go to the project tab ``News''.
\item Click on ``Submit''.
\item Add a subject, such as:
\begin{verbatim}
Windows release of FullSWOF_UI 1.01.00
\end{verbatim}
\item Add some details in the panel, such as:
\begin{verbatim}
The FullSWOF development team is pleased to announce
the release of the Windows package of FullSWOF_UI 1.01.00.
It contains the FullSWOF_UI jar file, the FullSWOF_1D and FullSWOF_2D 
exe files able to run under winxp, win7 32 bits and win 7 64 bits.
It is distributed for EVALUATION PURPOSE ONLY.
\end{verbatim}
\item Click on ``Submit''.
\end{enumerate}
\item Send an email to the list
\href{mailto:fullswof.infos@listes.univ-orleans.fr}{fullswof.infos@listes.univ-orleans.fr}, such as:
\begin{verbatim}
Title: FullSWOF_UI: Release 1.01.00 for Windows

Dear FullSWOF_UI enthusiasts,

We have just released a version of FullSWOF_UI 1.01.00 for Windows.
It can be downloaded here: <https://sourcesup.renater.fr/frs/?group_id=997>
(file FullSWOF_UI-1.01.00_win.zip)
It is designed to help Windows users to run FullSWOF on their 
exploitation system. It contains the FullSWOF_UI jar file, the FullSWOF_1D   
and FullSWOF_2D exe files able to run under winxp, win7 32 bits and 
win 7 64 bits.
It is distributed for EVALUATION PURPOSE ONLY.

For details see file README_win.txt .

Cheers.

The FullSWOF development team 
\end{verbatim}
\end{enumerate}

% % % % % % % % % % % % % % % % % % % % % % % % % % % % % % % % % % % % % % % %


\end{document}

