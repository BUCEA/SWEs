% !TEX encoding = UTF-8 Unicode
% !TEX spellcheck = en_US

\documentclass[a4paper, 11pt]{article}

\usepackage[utf8]{inputenc}
\usepackage[T1]{fontenc}
\usepackage{lmodern}
\usepackage[margin=2cm]{geometry}
\usepackage{enumitem}
\usepackage{fancyhdr}
\usepackage{lastpage}
\usepackage{amsmath}
\usepackage{graphicx}
\usepackage[hyphens]{url}
\usepackage{hyperref}



%% Specific commands
% the only place to specify version number (for consistency) - the date is the last modif for the code
\newcommand{\version}{1.04.10 (2014-02-05)}
\newcommand{\FullSWOF}{\emph{FullSWOF\_2D}}
\newcommand{\contactFullName}{Christian \textsc{Laguerre}, Fr\'ed\'eric \textsc{Darboux} and Carine \textsc{Lucas}}
\newcommand{\contactEmail}{ \href{mailto:christian.laguerre@math.cnrs.fr}{christian.laguerre@math.cnrs.fr}, \href{mailto:frederic.darboux@orleans.inra.fr}{frederic.darboux@orleans.inra.fr}, \href{mailto:carine.lucas@univ-orleans.fr}{carine.lucas@univ-orleans.fr}}
\newcommand{\MainWebSite}{\url{https://sourcesup.renater.fr/projects/fullswof-1d/}}
\newtheorem{cl}{Conclusion}

% Headers and footers
\lhead{\FullSWOF{} v\version}
\rhead{MAPMO --- INRA}
\cfoot{\thepage /\pageref{LastPage}}
\pagestyle{fancy}

% Lists: no spacing between items
\setlist{noitemsep}

\title{Improving the accuracy of floating-point computations
in~\FullSWOF{}.}

\author{\contactFullName\\ \contactEmail}
\date{2014-02-05} % Last modif for the note

\begin{document}

\maketitle

\thispagestyle{fancy}

This report contains a study of the influence of compiler options on the benchmark results.\\ 
It is similar to the report \texttt{Floating\_Points.pdf} (\url{https://sourcesup.renater.fr/docman/?group_id=877&view=listfile&dirid=1052}) which was performed on the software \emph{\textit{FullSWOF\_1D}}.
 Indeed, when running the 8~benchmarks of \emph{\textit{FullSWOF\_2D}} on several computers, we noticed differences in the results for one benchmark: the bump at rest. 
Performing some further testing, we also uncover some problems of symmetry: for example, inverting the geometry of the dam break from left-to-right to right-to-left did more than simply inverted the results.

These differences originated from approximations in the floating-point computations, and these approximations depended on the computer. After a diagnosis phase, various changes were made, both in the code and in the compiler options. In this report, we run \FullSWOF{} to determine the impact of these changes.

\section*{Testing \& Changes}

The first change we made was to replace each $x^2-y^2$ by $(x-y)(x+y)$. This modification affected the main part of the numerical scheme (see \texttt{scheme.cpp})
for the time evolution of the discharge and some fluxes. 
It is well-known that this way of computing reduces floating-point errors. 
For more details, we refer to \url{http://www.codeproject.com/Articles/29637/Five-Tips-for-Floating-Point-Programming}
and references therein. 

In order to solve symmetry problems, we had to use long doubles in some parts of the code. Namely, 
a computation with long double is now done in \FullSWOF{} in \texttt{hll.cpp}, \texttt{muscl.cpp}, \texttt{scheme.cpp}.
Moreover, the compilation option \texttt{-ffloat-store} was added. 

Let us study the influence of these changes on the results. In the following, we make a distinction between a 64-bit computer  and a 32-bit computer under Linux Ubuntu.
GNU C++ was used in both cases.

Only one out of the eight benchmarks differed depending on the float type and compilation option: the bump at rest. 
This test case involves very small perturbations of water height, and null velocities, so this may explain the difficulties to catch the analytic solution.

This study (see tables~\ref{Mac}-\ref{Linux}) shows that each long double is necessary to obtain results exactly identical to the reference solution. 
Moreover, it points out that the \texttt{-ffloat-store} option improves the result accuracy but roughly doubles the computation time. 
However, excepted for some specific points and output variables (especially the Froude number), most of differences were relatively small.

\section*{Conclusions \& Recommendations}

\begin{itemize}
\item There are still differences whether you run \FullSWOF{} under a 32-bit or a 64-bit operating system.

\item The option \texttt{-ffloat-store} helps in obtaining accurate results for very small water heights. This is useful in order to compare the numerical results with analytic solutions. 
But if you want to compare the numerical results with experimental results, you can remove this option to save computation time, especially if very shallow flows are not involved.
In particular, you should keep in mind that this option is very time consuming and you must balance the accuracy versus the computation time. 

\item Other improvements could be done. Do not hesitate to suggest modifications!

\item Currently, only the HLL and the MUSCL reconstruction have been tested and modified. Tests should also be done for the other fluxes and reconstructions. 
\end{itemize}


\begin{table}[htbp]
\begin{tabular}{ccccccc}
\hline
& & & & & Difference& Computation \\
Experiment & Long double & Long double & Long double & -ffloat&with reference& time (Mega\\
number & HLL & MUSCL & scheme & -store&  (bump at rest)  & clock ticks)  \\
& & & & & & \\
\hline
reference & yes & yes & yes &yes &&467\\
1 & no & yes & yes &yes & yes&428\\
2 & no & yes & yes &no & yes =1&192\\
3 & yes & no & yes &yes &no&470\\
4 & yes & no & yes &no &no&200\\
5 & yes & yes & no &yes &yes&444\\
6 & yes & yes & no &no &yes =5&199\\
7 & yes & yes & yes &no &no&197\\
\hline
\end{tabular}
\caption{Linux - 64 bits}
\label{Mac}
\end{table}

\begin{table}[htbp]
\begin{tabular}{ccccccc}
\hline
& & & & & Difference& Computation \\
Experiment & Long double & Long double & Long double & -ffloat&with reference& time (Mega\\
number & HLL & MUSCL & scheme & -store&  (bump at rest)  & clock ticks)  \\
& & & & & & \\
\hline
reference & yes & yes & yes &yes && 977\\
1 & no & yes & yes &yes & yes& 919\\
2 & no & yes & yes &no & yes& 520\\
3 & yes & no & yes &yes &no&987\\
4 & yes & no & yes &no &yes& 482\\
5 & yes & yes & no &yes &yes& 964\\
6 & yes & yes & no &no &yes&457\\
7 & yes & yes & yes &no &yes =6& 465\\
\hline
\end{tabular}
\caption{Linux - 32 bits}
\label{Linux}
\end{table}
\end{document}

