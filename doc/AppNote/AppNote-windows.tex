% !TEX encoding = UTF-8 Unicode
% !TEX spellcheck = en_US

\documentclass[a4paper, 11pt]{article}

\usepackage[utf8]{inputenc}
\usepackage[T1]{fontenc}
\usepackage{lmodern}
\usepackage[margin=2cm]{geometry}
\usepackage{enumitem}
\usepackage{fancyhdr}
\usepackage{lastpage}
\usepackage{hyperref}

% Specific commands
\newcommand{\FullSWOFnD}{\emph{FullSWOF\_1D/2D}}
\newcommand{\FullSWOFoneD}{\emph{FullSWOF\_1D}}
\newcommand{\FullSWOFtwoD}{\emph{FullSWOF\_2D}}
\newcommand{\SWASHES}{\emph{SWASHES}}
\newcommand{\contactFullName}{Frédéric \textsc{Darboux}}
\newcommand{\contactEmail}{\href{mailto:Frederic.Darboux@orleans.inra.fr}{Frederic.Darboux@orleans.inra.fr}}
\newcommand{\MainWebSiteFSoneD}{\url{https://sourcesup.renater.fr/projects/fullswof-1d/}}
\newcommand{\MainWebSiteFStwoD}{\url{https://sourcesup.renater.fr/projects/fullswof-2d/}}
\newcommand{\MainWebSiteSWASHES}{\url{https://sourcesup.renater.fr/projects/swashes/}}

% Headers and footers
\lhead{Running \FullSWOFnD{} or \SWASHES{} under windows}
\rhead{INRA-UR272-Sols --- MAPMO}
\cfoot{\thepage /\pageref{LastPage}}
\pagestyle{fancy}

% Lists: no spacing between items
\setlist{noitemsep}

\title{Application note:\\
Using Cygwin to compile and run\\
\FullSWOFoneD, \FullSWOFtwoD{} or \SWASHES{}\\
under windows.}

\author{\contactFullName, \contactEmail}
\date{2013-09-04}

\begin{document}

\maketitle

\thispagestyle{fancy}

\FullSWOFoneD, \FullSWOFtwoD{} and \SWASHES{} have been developed under Unix-like environments. Although it may not be required, it is convenient to use such a Unix-like environment under windows. This application note gives directions about installing the Unix-like environment Cygwin, and using it to compile and run  \FullSWOFoneD, \FullSWOFtwoD{} and \SWASHES{}.

\section{Installation of Cygwin}

\begin{enumerate}
\item From \url{www.cygwin.com}, download the file \textbf{setup.exe} and save it into a dedicated directory (e.g. \verb!c:\software\cygwin_install!).
\item Launch \textbf{setup.exe}.
\item Click on the ``next'' button on the first screens. Then, choose a download site (next to you) and click on the ``next'' button once more. You will be prompted to select packages. A basic set of packages are already selected. To compile the software smoothly, you need to add a few more packages by clicking on ``skip'':
	\begin{itemize}
	\item from the category ``Archive''
		\begin{itemize}
		\item unzip: Info-ZIP decompression utility
		\end{itemize}
	\item from the category ``Devel''
		\begin{itemize}
		\item gcc-g++: C++ compiler 
		\item make: The GNU version of the 'make' utility
		\end{itemize}
	\end{itemize}
Additionally, if you want to display graphs using gnuplot, you also need:
	\begin{itemize}
	\item from the category ``Graphics''
		\begin{itemize}
		\item gnuplot: A command-line driven interactive function plotting utility
		\end{itemize}
	\item from the category ``X11''
		\begin{itemize}
		\item xorg-server: X.org X servers
		\item xinit: X.org X server initializer
		\end{itemize}
	\end{itemize}
\item Then click ``next'' up to the start of downloading.
\item Finally, click ``End''.
\end{enumerate}

This will use about of  250~MB (490~MB with gnuplot).

\section{Compiling and running the software}

\begin{enumerate}
\item Download the \FullSWOFoneD\footnote{\MainWebSiteFSoneD}, \FullSWOFtwoD{}\footnote{\MainWebSiteFStwoD} or \SWASHES{}\footnote{\MainWebSiteSWASHES} package into a dedicated directory.
\item Open a Cygwin terminal using the desktop icon or the menu.
\item Move to your dedicated directory. For example, to access the directory \verb!D:\user\code!, you should enter the command \verb!cd /cygdrive/d/user/code!
\item Unzip the package (e.g. \verb!unzip package.zip!).
\item Move to the newly-created directory. This directory will contain all the files related to the source code.
\item Refer to the software-specific documentation to compile and run the code.
\item To use gnuplot, open the cygwin terminal and type \verb!startx!. This will launch the X server. Then type \verb!gnuplot!.
\end{enumerate}

\emph{For more information about Cygwin, see \url{http://www.cygwin.com}}

\end{document}
